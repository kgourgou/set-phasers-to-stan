\documentclass{article}\usepackage[]{graphicx}\usepackage[]{color}
%% maxwidth is the original width if it is less than linewidth
%% otherwise use linewidth (to make sure the graphics do not exceed the margin)
\makeatletter
\def\maxwidth{ %
  \ifdim\Gin@nat@width>\linewidth
    \linewidth
  \else
    \Gin@nat@width
  \fi
}
\makeatother

\definecolor{fgcolor}{rgb}{0.345, 0.345, 0.345}
\newcommand{\hlnum}[1]{\textcolor[rgb]{0.686,0.059,0.569}{#1}}%
\newcommand{\hlstr}[1]{\textcolor[rgb]{0.192,0.494,0.8}{#1}}%
\newcommand{\hlcom}[1]{\textcolor[rgb]{0.678,0.584,0.686}{\textit{#1}}}%
\newcommand{\hlopt}[1]{\textcolor[rgb]{0,0,0}{#1}}%
\newcommand{\hlstd}[1]{\textcolor[rgb]{0.345,0.345,0.345}{#1}}%
\newcommand{\hlkwa}[1]{\textcolor[rgb]{0.161,0.373,0.58}{\textbf{#1}}}%
\newcommand{\hlkwb}[1]{\textcolor[rgb]{0.69,0.353,0.396}{#1}}%
\newcommand{\hlkwc}[1]{\textcolor[rgb]{0.333,0.667,0.333}{#1}}%
\newcommand{\hlkwd}[1]{\textcolor[rgb]{0.737,0.353,0.396}{\textbf{#1}}}%

\usepackage{framed}
\makeatletter
\newenvironment{kframe}{%
 \def\at@end@of@kframe{}%
 \ifinner\ifhmode%
  \def\at@end@of@kframe{\end{minipage}}%
  \begin{minipage}{\columnwidth}%
 \fi\fi%
 \def\FrameCommand##1{\hskip\@totalleftmargin \hskip-\fboxsep
 \colorbox{shadecolor}{##1}\hskip-\fboxsep
     % There is no \\@totalrightmargin, so:
     \hskip-\linewidth \hskip-\@totalleftmargin \hskip\columnwidth}%
 \MakeFramed {\advance\hsize-\width
   \@totalleftmargin\z@ \linewidth\hsize
   \@setminipage}}%
 {\par\unskip\endMakeFramed%
 \at@end@of@kframe}
\makeatother

\definecolor{shadecolor}{rgb}{.97, .97, .97}
\definecolor{messagecolor}{rgb}{0, 0, 0}
\definecolor{warningcolor}{rgb}{1, 0, 1}
\definecolor{errorcolor}{rgb}{1, 0, 0}
\newenvironment{knitrout}{}{} % an empty environment to be redefined in TeX

\usepackage{alltt}
\usepackage[utf8]{inputenc}
\usepackage[english]{babel}
\usepackage{hyperref}
\hypersetup{
    colorlinks=true,
    linkcolor=blue,
    filecolor=magenta,
    urlcolor=cyan,
}

\urlstyle{same}

\usepackage{libertine}


\title{Homework 5: working with Stan}
\author{Kostis, Steve, Jie}
\date{\today}
\IfFileExists{upquote.sty}{\usepackage{upquote}}{}
\begin{document}

\maketitle

Stan is a probabilistic programming language that can be used to estimate the parameters that generated observed data. To see this, you will use Stan to estimate the parameters you chose in generating the data for Homework 4 using the Stochastic Simulation Algorithm (SSA).

\section{Required}

\begin{enumerate}
    \item Write a Stan script that will estimate the model parameters from the SSA section of homework 4.
    \item Write a script (R, Python, MATLAB, etc.) that loads the data from the
      SSA (either stored or simulated within the script), sets up the relevant
      dictionaries and calls the stan function. For this, we provide skeleton
      scripts, in case you don't want to write your own. 
\end{enumerate}

Then, run the estimation for various values of $\alpha$ and $\mu$. How accurate
is the estimation when $\alpha/\mu\simeq 1$?  What about $\alpha/\mu\gg 1$ or
$\alpha/\mu\ll 1$?

\begin{itemize}
\item Hint 1: check out \url{https://github.com/kgourgou/set-phasers-to-stan/tree/homework} and look at {\tt 8schools} and {\tt regression-example} for examples of how to write Stan and R scripts.

\item Hint 2: in the {\tt homework} folder of the github page, you will find skeleton scripts for Stan, R, and Python.

\item Hint 3: if you were unable to generate the data for homework 4, email ``slauer'' at ``schoolph.umass.edu'' and we will send you the script to do so.
\end{itemize}

\section{Optional}

\begin{enumerate}
    \item Write a .stan file and a script that will estimate the model parameters from the Chemical Langevin Equation section of homework 4.
    \item Write a Stan and API script that will estimate the model parameters
      from the Reaction Rate Equations of homework 4.
\end{enumerate}

We have not done these yet - you're on your own! ODEs are formulated differently
than the models in our examples and for the other parts of this homework. It may
help to study the ``Solving Differential Equations'' section of the Stan manual (p.191, this can also befound on the github page).

\end{document}
